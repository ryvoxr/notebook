\documentclass{article}
\usepackage{natbib}
\linespread{2}
\title{Rust: A Richly Typed, Memory-safe, Algebraic Systems Language for Writing Perfect Code}
\author{Henry Oehlrich}

\begin{document}
\maketitle{}

\section{Borrow Checker}

Rust approaches memory management with a new paradigm. It uses neither a
garbage collector nor forces memory management on the programmer; instead it
achieves memory safety using a borrow checker. The borrow checker has two
rules: data has one owner and data may have multiple readers or one writer.
When a piece of non-global data is instantiated, a sized portion of data from
the stack, or a chunk of data in the heap, is allocated to it. The variable
that the data is assigned to is its owner. Passing data by value (IE not
through a pointer) moves the data to the function it was passed into. This data
is gone, it can neither be accessed nor modified. In order to share data
without moving, pass a reference to the value.

\cite{rustforrustaceans} \\
\cite{rust-lang.org} \\
\cite{rust-by-example} \\
\cite{the-c-programming-language} \\
\cite{noboilerplate}

\bibliographystyle{mla}
\bibliography{refs} 

\end{document}
