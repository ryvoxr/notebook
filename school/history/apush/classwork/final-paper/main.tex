\documentclass{article}

\title{APUSH Final Paper}
\author{Henry Oehlrich}

\begin{document}
\maketitle{}

Thesis: The Great Depression was primarily caused by the unsustainable credit
expansion and subsequent boom-bust cycle fueled by the Federal Reserve. The
loose monetary policies and artificially low interest rates of the 1920s led to
excessive credit creation, bad investments, and misallocations of resources,
creating an unsustainable economic boom. The subsequent contractionary monetary
policies pursued by central banks further deepened the economic downturn,
hindering the necessary market adjustments and impeding the natural recovery
process. 

The stock market crash in October of 1929 and the following depression of the
1930's are often used as examples of failures of capitalism and unregulated
markets. It is not proven that the stock market crash mandated the mass
unemployment that followed. It instead could be argued that the government's
actions in the 20s and response to the downturn in the 30s were what caused the
catastrophe that was the Great Depression. The Great Depression was not created
by a failure of business; instead, it was created by a failure of government.
The strength and length of the Depression was produced from artificial
lengthening of the boom, interventionist response policies, and the failure of
the Federal Reserve to act as a "lender of last resort".

\section{Easy money of the late 1920's}


The stock market crash in October of 1929 succeeded several years of
considerable credit expansion by the Federal Reserve System. After business had
a downturn in 1924, the Federal Reserve created \$500 million of new credit.
This expansion of credit initially appeared positive, creating a boom that
overwrote the downturn in 24. In 1927, the Federal Reserve once again created
credit, further growing the money supply. These expansions of credit propelled
the (otherwise free market) economy to the late 20's. As the expansion matured,
profits began to fall, business costs increased, and interest rates began to
rise. The Fed, fearing runaway inflation, stopped the cash flow. Within the
first few months of 1929, the Federal Reserve forced interest rates up and led
the country into the crash and following depression.

\section{Interventionist response policies}

In October of 1929, the stock market crashed. Although unemployment rose
immediately after the crash, the unemployment rate peaked at 9\% two months
after the crash, and then began to trend downward. By June of 1930, 8 months
later, unemployment was down to 6\%. By recession standards for the time, this
was not extraordinary. In 1920 there was a similar downturn and unemployment
hit 11\% in 8 months before recovering quickly. The difference between these
two downturns was government intervention. President Woodrow Wilson and Warren
Harding shared the depression of the early 20s. Both Wilson and Harding did
nothing to remedy the economy and it recovered quickly. On the other hand
Presidents Hoover and Roosevelt engaged in large amounts of government spending
in an attempt to stimulate the economy. After a series of unprecedented
government interventions, unemployment rate shot to over 20\% for 3 years
straight.

In early June of 1930, unemployment was at 6\% and trending downward. The
government, feeling compelled to do something, passed into law the Smoot-Hawley
Tariffs. The tariffs were designed to reduce unemployment by restricting
imports to the United States so that more goods were produced domestically by
American workers. Although well meaning, it had disastrous effects. 1008
economists signed a public appeal to not pass the bill stating that the bill
would not reduce unemployment [citation needed]. Instead, the economists
argued, it would lead to retaliation that would make it harder for Americans to
sell their goods in other countries. As predicted, by imposing large tariffs,
the Act sparked retaliation from other nations, leading to a decline in
international trade. Dartmouth economist Douglas Irwing wrote that the tariff
has "become synonymous with an avalanche of protectionism that led to the
collapse of world trade and the Great Depression" [citation needed]. In the
following four months after passing the tariffs, unemployment rose to 14\%.
Unemployment stayed in the double digits until the end of the depression in
1939.

Herbert Hoover was president at the start of the Great Depression. Hoover,
having previously been the Secretary of Commerce, believed that the stock
market crash and the downturn as a whole was caused by overspeculation of
investors buying on credit. In an attempt to prevent panic from spreading
throughout the economy, he told business leaders to maintain wages and
collected money from private business in order to stimulate the economy. He
believed that financial losses should affect profits, not employment. His
response was nearly entirely based on spending copious amounts of money in
order to stimulate the economy. Government and business had spent more in the
first six months of 1930 than in the entire previous year. [citation needed]
Although the efficacy of Hoover's numerous remedies was limited, Hoover's
defenders praise his attempts to fix the economy. The Herbert Hoover
Presidential Library and Museum defends Hoover stating that "no one in his
place could have done more" and that "very few of his predecessors have done as
much."

\section{Failure of the Federal Reserve}

The Federal Reserve was established in 1913 in response to financial panics in
the late 19th and early 20th century. Its duty was to act as a lender of last
resort during times of financial stress. It would (theoretically) accomplish
this by providing liquidity, or cash, to banks. The Fed was designed to stabilize
the financial system and prevent widespread bank runs.

In their book \textit{A Monetary History of the United States}, Milton Friedman
and Anna Schwartz propose another explanation for the Great Depression. They
determine that the Depression was ultimately caused by a deflation of the money
supply from the failure of the Federal Reserve to save the banking system. From
1929 to 1933, one third of the nations banks (around 7,500) closed due to
failure or merger. Resulting from this, the nations money supply, defined as
bank deposits combined with currency circulation, shrank by a third. The irony,
Friedman and Schwartz argue, is that the Federal Reserve was created to serve
the very purpose it declined to serve.

In the early years of the Great Depression, the Federal Reserve enacted policy
that supported monetary contraction. Instead of providing sufficient money to
the banking system, the Fed restricted the money supply and raised interest
rates. These actions further contracted the economy and exacerbated the banking
crisis. By not acting as a lender of last resort, the Federal Reserve failed to
provide the necessary money to stabilize the financial system and prevent
widespread bank failures.

As the depression progressed, there were numerous bank runs. Panicked
depositors, now unsure of the safety of their money, rushed to withdraw their
money from banks. These runs caused banks to experience severe shortages,
leading to bank failures. The Federal Reserve, as the lender of last resort,
had the authority, ability, and duty to provide emergency loans to troubled
banks to prevent the crisis. However, the Fed's response was inadequate; it did
not provide sufficient money to overrun banks in need.

\end{document}
