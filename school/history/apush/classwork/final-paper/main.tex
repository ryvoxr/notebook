\documentclass{article}

\title{APUSH Final Paper}
\author{Henry Oehlrich}

\begin{document}
\maketitle{}

The stock market crash in October of 1929 and the following depression of the
1930's are often used as examples of failures of capitalism and unregulated
markets. It is not proven that the stock market crash mandated the mass
unemployment that followed. It instead could be argued that the government's
actions in the 20s and response to the downturn in the 30s were what caused the
catastrophe that was the Great Depression. The Great Depression was not created
by a failure of business; instead, it was created by a failure of government.
The strength and length of the Depression was produced from artificial
lengthening of the boom, interventionist response policies, and the failure of
the Federal Reserve to act as a "lender of last resort".

In October of 1929, the stock market crashed. Two months later unemployment
peaked at 9\%. By June of 1930, 8 months later, unemployment was down to 6\%.
By recession standards for the time, this was not extraordinary. In 1920 there
was a similar downturn and unemployment hit 11\% in the same 8 months before
recovering. The difference between these two downturns was government
intervention. President Woodrow Wilson and Warren Harding shared the depression
of the early 20s. Both Wilson and Harding did nothing to remedy the economy and
it recovered quickly. On the other hand Presidents Hoover and Roosevelt engaged
in large amounts of government spending in an attempt to stimulate the economy.

In early June of 1930, unemployment was at 6\% and dropping. The government,
feeling compelled to do something, passed into law the Smoot-Hawley Tariffs.
The tariffs were designed to reduce unemployment by restricting imports to the
United States so that more goods were produced domestically by American
workers. Although well meaning, it had disastrous effects. 1008 economists
signed a public appeal to not pass the bill stating that the bill would not
reduce unemployment. Instead it would lead to retaliation that would make it
harder for Americans to sell their goods in other countries. In the following
four months after passing the tariffs, unemployment rose to 14\%. Unemployment
stayed in the double digits until the end of the depression in 1939.

\section{Failure of the Federal Reserve}

\section{Interventionist response policies}
- Smoot-Hawley Tariff
- New Deal
- NRA, AAA, and Wagner Act

\end{document}
