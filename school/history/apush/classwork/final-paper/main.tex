\documentclass[12pt]{article}
\usepackage{geometry}
\usepackage{setspace}
\usepackage{epigraph}
\usepackage[notes,natbib,isbn=false,backend=biber]{biblatex-chicago}  

\doublespacing
\geometry{
    a4paper,
    margin=1.1in,
}
\bibliography{refs.bib}

\title{Causes of The Great Depression}
\author{Henry Oehlrich}

\begin{document}
\maketitle{}

The stock market crash in October of 1929 and the following depression of the
1930’s are often used as examples of failures of capitalism and unregulated
markets. It is not proven that the stock market crash mandated the mass
unemployment that followed. It instead could be argued that the government’s
actions in the 20’s and response to the downturn in the 30’s were what caused
the catastrophe that was the Great Depression. The Great Depression was
primarily caused by the unsustainable credit expansion and subsequent boom-bust
cycle fueled by the Federal Reserve. Artificially low interest rates of the
1920's led to excessive credit creation, bad investments, and misallocations of
resources, creating an unsustainable economic boom. The contractionary
monetary policies pursued by central banks after the stock market crash in 1929
further deepened the economic downturn, preventing necessary market adjustments
from occurring and stopping the natural recovery process.

\section{Easy money of the late 1920's}

The 1920's, often referred to as the Roaring Twenties, contained a period of
apparent prosperity and economic growth in the United States. Unfortunately,
this prosperity and booming economic activity was largely driven by
unsustainable factors. The artificial boom of the 1920's was caused by
government policies, particularly monetary interventions.

One of the main causes of the boom of the 1920's was the expansionary monetary
policies of Federal Reserve. The Fed’s decision to keep interest rates
artificially low led to an increase in bank credit and the availability of
money. For example, from 1921 to 1929, the money supply in the United States
expanded by 60\%\autocite{fredmoneysupply}. This influx of easy credit,
nicknamed ”Easy money”, stimulated economic activity and contributed to the
appearance of a booming economy.

The availability of loans, due to low interest rates, and the optimistic
outlook of investors, due to a false boom, fueled speculative excesses,
particularly in the stock market. This resulted in a rapid increase in stock
prices and trading volumes, with the Dow Jones Industrial Average reaching
unprecedented levels. For example, between 1924 and 1929, the Dow rose by over
200\%\autocite{macrotrendsdow} This overspeculation created a bubble and set
the stage for a massive stock market crash.

The artificial boom of the 1920's was further propelled by excessive consumer
spending and reliance on debt. Readily available loans and installment buying
(going into debt to purchase something) allowed individuals to engage in higher
levels of consumption. Because of the volume of credit, much of this spending
was financed by credit rather than real savings. For example, consumer debt
rose from \$1.8 billion in 1920 to \$7 billion in
1929\autocite{americaneconomy20s}. This unsustainable consumer debt burden left
people and assets vulnerable when the economic downturn hit.

Due to the fundamental laws of economics (i.e. business cycles) no boom is
indefinite and all booms end in a bust. The issue is that artificially large
booms tend to arrange themselves with egregiously large busts. This was shown
to be true in October of 1929 on Black Monday when the Dow dropped 13\% in one
day and again on Black Tuesday when the Dow dropped another 12\%
\autocite{stockmarketcrash}. The decrease continued unfaltering until the
summer of 1932 where the Dow was 89\% below its peak back in 1929.

\section{Interventionist response policies}

In October of 1929, the stock market crashed. Although unemployment rose
immediately after the crash, the unemployment rate peaked at 9\% two months
after the crash, and then began to trend downward. By June of 1930, 8 months
later, unemployment was down to 6\%\autocite{nber}. By recession standards for
the time, this was not extraordinary. In 1920 there was a similar downturn and
unemployment hit 11\%\autocite{nber} in 8 months before recovering quickly. The
difference between these two downturns was government intervention. President
Woodrow Wilson and Warren Harding shared the depression of the early 20's. Both
Wilson and Harding did nothing to remedy the economy and it recovered quickly.
On the other hand Presidents Hoover and Roosevelt engaged in large amounts of
government spending in an attempt to stimulate the economy. After a series of
unprecedented government interventions, unemployment rate shot to over
20\%\autocite{nber} for 3 years straight.

In early June of 1930, unemployment was at 6\% and trending downward. The
government, feeling compelled to do something, passed into law the Smoot-Hawley
Tariffs. The tariffs were designed to reduce unemployment by restricting
imports to the United States so that more goods were produced domestically by
American workers. Although well meaning, it had disastrous effects. 1008
economists signed a public appeal to not pass the bill stating that the bill
would not reduce unemployment\autocite{economistssmoothawley}. Instead, the
economists argued, it would lead to retaliation that would make it harder for
Americans to sell their goods in other countries. As predicted, by imposing
large tariffs, the Act sparked retaliation from other nations, leading to a
decline in international trade. Dartmouth economist Douglas Irwing wrote that
the tariff has ”become synonymous with an avalanche of protectionism that led
to the collapse of world trade and the Great Depression”
\autocite{peddlingprotectionism}. In the following four months after passing
the tariffs, unemployment rose to 14\%\autocite{nber}. Unemployment stayed in
the double digits until the end of the depression in 1939.

Herbert Hoover was president at the start of the Great Depression. Hoover,
having previously been the Secretary of Commerce, believed that the stock
market crash and the downturn as a whole was caused by overspeculation of
investors buying on credit. In an attempt to prevent panic from spreading
throughout the economy, he told business leaders to maintain wages and
collected money from private business in order to stimulate the economy. He
believed that financial losses should affect profits, not employment. His
response was nearly entirely based on spending copious amounts of money in
order to stimulate the economy. Government and business had spent more in the
first six months of 1930 than in the entire previous year.\autocite{hhplm} 
Although the efficacy of Hoover’s numerous remedies was limited, Hoover’s
defenders praise his attempts to fix the economy. The Herbert Hoover
Presidential Library and Museum defends Hoover stating that ”no one in his
place could have done more” and that ”very few of his predecessors have done as
much.”

\section{Failure of the Federal Reserve System}

The Federal Reserve was established in 1913 in response to financial panics in
the late 19th and early 20th century. Its duty was to act as a lender of last
resort during times of financial stress. It would (theoretically) accomplish
this by providing liquidity, or cash, to banks. The Fed was designed to
stabilize the financial system and prevent widespread bank runs.

In their book \textit{A Monetary History of the United States}, Milton Friedman
and Anna Schwartz propose an explanation for the Great Depression
\autocite{amonetaryhistory}. They determine that the Depression was ultimately
caused by a deflation of the money supply from the failure of the Federal
Reserve to save the banking system. From 1929 to 1933, one third of the nations
banks (around 7,500) closed due to failure or merger \autocite{nber}. Resulting
from this, the nations money supply, defined as bank deposits combined with
currency circulation, shrank by a third. The irony, Friedman and Schwartz
argue, is that the Federal Reserve was created to serve the very purpose it
declined to serve.

In the early years of the Great Depression, the Federal Reserve enacted policy
that supported monetary contraction. Instead of providing sufficient money to
the banking system, the Fed restricted the money supply and raised interest
rates. These actions further contracted the economy and exacerbated the banking
crisis. By not acting as a lender of last resort, the Federal Reserve failed to
provide the necessary money to stabilize the financial system and prevent
widespread bank failures.

As the depression progressed, there were numerous bank runs. Panicked
depositors, now unsure of the safety of their money, rushed to withdraw their
money from banks. These runs caused banks to experience severe shortages,
leading to bank failures. The Federal Reserve, as the lender of last resort,
had the authority, ability, and duty to provide emergency loans to troubled
banks to prevent the crisis. However, the Fed’s response was inadequate; it did
not provide sufficient money to overrun banks in need.


The Great Depression remains in history as the single worst economic decline
ever in United States. Examining the causes of the Great Depression and
government's response provides valuable insights into this significant economic
event. The combination of an artificial boom in the 1920s followed by a
contractionary and interventionist government response after the stock market
crash, contributed to the severity and duration of the economic downturn.

\setlength{\epigraphwidth}{0.8\textwidth}
\epigraph{"I would like to say to Milton and Anna: Regarding the Great Depression, you’re
right. We did it. We’re very sorry."}{\textit{Ben Bernanke (Former Federal Reserve Governor)}}


\newpage
\printbibliography
\end{document}
