\documentclass{article}
\usepackage[notes,natbib,isbn=false,backend=biber]{biblatex-chicago}  
\usepackage{setspace}

\doublespacing
\bibliography{refs.bib}

\author{Henry Oehlrich}
\title{Economic Unintended Consequences}

\begin{document}

\autocite{merton1936}
\autocite{hayek1945}
\autocite{econlib2023}
\autocite{reason2019}
\autocite{reason2023}
\autocite{fda2023}
\autocite{intelligencer2020}
\autocite{wired2022}
\autocite{simpleflying2020}
\autocite{usdot2014}
\autocite{reason2018}

\newpage

https://henryoehlrich.xyz/unintended-consequences.mp3

\section{Introduction}

Imagine you are part of an alien species tasked with studying human life. Your
job is to determine to what extent humans can foresee the consequences of their
own decisions. What follows are the highlights of your research.

\subsection{Case One: Taxing Light and Air}

Year: 1696
Location: Great Britain
Confidence in human foresight: undetermined

The problem: Britain needs money

The solution: Introduce a property tax based on the number of windows in a
house. This tax aimed to be a non-intrusive way to tax wealthier people more.

Adam Smith even approved of the Window Tax saying that it was "inoffensive
because its assessment did not require the assessor to enter the residence".

Sounds like a great, well intentioned idea! What could possibly go wrong?

While wealthier individuals tended to have more windows, the tax actually had
more impact on the poorest people living in tenements with numerous windows. To
evade the tax, tenement landlords bricked up windows in their buildings. This
led to a drastic reduction in the already poor safety, hygiene, and quality of
living in tenements.

\subsection{Case Two: Prohibition}

Year: 1919
Location: The United States of America
Confidence in human foresight: moderately low

The problem: alcohol is ruining the moral fiber of America and is the
fundamental reason behind problems like poverty and domestic violence.

The solution: pass the 18th Amendment to the Constitution banning the
manufacture, sale, and transfer of all alcoholic beverages.

Sounds like a great, well intentioned idea! What could possibly go wrong?

It turns out, everybody still wanted alcohol. Instead of rolling over and
staying sober, Americans found new ways to get booze.

Organized crime groups established a lucrative black market in producing,
smuggling, and selling illicit liquor. These crime groups worked side-by-side
with corrupt policemen and government officials. Thousands of illegal bars,
known as speakeasies began serving alcohol.

Some people tried to make alcohol at home for their own consumption. This had the
disastrous consequence of poorly made, home-brewed alcohol that contained the
poison methanol. Thousands of Americans died during prohibition from methanol
poisoning alone.

\subsection{Case Three: The CAFE Standards Oopsie}

Year: 1975
Location: The United States of America
Confidence in human foresight: poor

The problem: cars are using too much fossil fuels and the resulting carbon emissions
are probably killing the planet

The solution: Enact fleet-wide energy consumption average requirements called
the CAFE Standards.

These standards award fines to manufacturers who fail to meet the efficiency
requirements. The standards are categorized by vehicle size with stricter
requirements for cars than for trucks.  

Sounds like a great, well intentioned idea! What could possibly go wrong?

It turns out, manufacturers don't like paying fines. 

Instead of designing and manufacturing more expensive and fuel efficient cars,
manufacturers simply switched to producing larger fuel-guzzling SUV's that meet
the legal definition for a "light truck". These "light trucks" suffered from
far lower fuel requirements than cars. Now, instead of a more efficient fleet,
the average fuel efficiency of vehicles on the road actually decreased.


\subsection{Case Three: Hurricane Havoc Hospitality}

Year: 2005
Location: Gulf Coast, United States
Confidence in human foresight: vanishingly small

The problem: Hurricane Katrina has devastated the Gulf Coast, displacing countless
residents in need of shelter

The solution: Enact price gouging laws to protect consumers from exorbitant
hotel rates during times of crisis

Sounds like a great, well intentioned idea! What could possibly go wrong?

This regulation posed unforeseen challenges for hotels. Faced with the surge in
demand for accommodations, establishments struggled to cover their increased
operational costs. The cap on room rates, coupled with the increased expenses
due to emergency preparations and potential property damage created a financial
dilemma for hotels.

The reduced incentive for hotels to operate during the crisis led to closures
and reduction of services. As a result, the shortage of available shelter for
displaced people grew even larger.

\subsection{Case Four: Covid Ghost Planes}

Year: 2020
Location: Most developed nations
Confidence in human foresight: near zero

Airports use a slotting system to determine how many flights airlines are
allowed to fly.

The problem: There are not enough slots in major airports and some airlines are
holding slots they don't need to prevent other airlines from competing with
them.

The solution: Enact a "use them or lose them" policy that requires airlines to
utilize their slots at 80\% capacity. If airlines don't, the FAA will require
them to relinquish their slots (most likely to their direct competitors).

Sounds like a great, well intentioned idea! What could possibly go wrong?

It turns out, airlines don't like losing their slots. During Covid, air travel
was at an all time low. Despite this, airlines still flew at the required 80%
capacity in order to keep their slots. This resulted in countless "ghost
flights" where planes would take off, fly, and land at their destination
carrying few or zero passengers.

\subsection{Case Five: Sesame Seed Shenanigans}

Year: 2021
Location: The United States of America
Confidence in human foresight: non-existent


The problem: People are allergic to sesame seeds and need to know what products
contain them

The solution: Add sesame seeds to the Food and Drug Administration's list of
items that are required to be declared in food. Then, slam companies who don't
follow the rules with massive fines.

Sounds like a great, well intentioned idea! What could possibly go wrong?

It turns out companies don't like to pay fines. Instead of spending millions of
dollars in order to test their products for sesame seeds, companies simply
sprinkled in a few sesame seeds to products that previously contained none and
then marked "contains sesame seeds" on the box. Over the next few months, the
number of sesame-seed-free products plummeted.

"I don't think anyone envisioned there being a decrease in the availability of
products that are safe choices for sesame allergic consumers" said the FDA
head. Except perhaps the many food companies that responded to the FDA's new
policy by doing exactly this. What he means, of course, is that nobody within
the FDA could have predicted this outcome.

\subsection{Conclusion}

In unraveling the mysteries of human decisions, a constant thread is the good
intention behind each action. Yet, these case studies show the vulnerability of
good intentions to the impossibly complicated nature of human behavior.
Unintended consequences challenge the purity of initial motives. As we, alien
observers, reflect on the journey, it becomes clear that understanding the
relationship between intentions and outcomes is a crucial aspect of
comprehending the complex tapestry of human existence.

\newpage

\printbibliography

\end{document}

