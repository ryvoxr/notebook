\documentclass{article}
\title{Assessing the Economy}
\author{Henry Oehlrich}

\usepackage[margin=1.5in]{geometry}

\begin{document}
\maketitle

\subsubsection*{According to the Fed website's data, what phase does the U.S. Economy seem to be in? What evidence points to this interpretation?}

According to the Fed website's data, the U.S. Economy seems to be at the end of
a boom. The unemployment rate has stopped decreasing and is leveled out. When
looking at all other unemployment troughs, the next thing that happens is the
rate skyrockets.

\subsubsection*{According to the Bureau of Economic Analysis data, what phase does the U.S. Economy seem to be in? Evidence?}

According to the Bureau of Economic Analysis data, the U.S. Economy seems to be
in a boom. The GDP is increasing, albeit at a decreasing rate. Q2 of 2023 saw a
2\% increase in GDP from Q1 of 2023.

\subsubsection*{Take a look at Macrotrends Dow Jones Industrial Average. What phase did the U.S. Economy hit in Spring 2020-and why was that?}

In Spring 2020, the U.S. Economy hit a major recession. This was due to the
COVID-19 pandemic.

\subsubsection*{What phase did the U.S. Experience the following year? Evidence?}

The economy experienced a boom. The Dow Jones Industrial Average increased by
18\% from the bottom of the recession to the end of the year.

\subsubsection*{View the Bureau of Labor Statistics chart of various prices. What phase do the two shaded areas represent?}

The shaded areas represent a recession.

\subsubsection*{Now look at today's overall prices. Where are we on the business cycle roller-coaster today? What evidence suggests this?}

Today we are at the end of a Keynesian boom. Price are still increasing but at
a decreasing rate. Disregarding the current boom, inflation is still the
highest it has been since 2008. 

\end{document}
