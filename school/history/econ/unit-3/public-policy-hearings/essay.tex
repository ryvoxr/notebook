\documentclass[12pt]{article}
\usepackage{geometry}
\usepackage{setspace}
\usepackage{epigraph}
\usepackage[notes,natbib,isbn=false,backend=biber]{biblatex-chicago}

\doublespacing
\geometry{
    a4paper,
    margin=1in,
    top=1in,
}
\bibliography{refs.bib}

\title{A Deficit-Neutral Tax Reform Proposal}
\author{Henry Oehlrich \and Laith Alia \and Sabe Harrison}
\date{}

\begin{document}

\maketitle

The tax code of the United States has become a tangled mess of deductions,
credits, exemptions, and loopholes that affect taxpayers unequally and grant
tremendous sums of money to politically favored industries. These provisions
should be eliminated and replaced with a simple, transparent, and fair tax code
that demands the same tax bill from those making the same income. We believe
that income should be taxed once and at the lowest rates possible in order to
allow for greater individual discretion and economic growth.

The loopholes that litter the tax code disproportionately benefit the rich.
Rich Americans have both the means and incentive to hire expensive accountants
to minimize their tax burden. These accountants comb through the dense
paragraphs finding loopholes and exemptions that save their clients vast
amounts of money every year. Unfortunately, poor and middle class families do
not have the same ability to do so. This disparity results in poor and middle
class taxpayers being required to fulfill their entire tax base and rich
taxpayers making nonproductive investments in order to lower their effective
rate. By removing unequal tax loopholes, the code will be stronger, more fair,
and more accessible.

A common misconception about taxes is that the rich will pay more if tax rates
are higher. This has been proven to be false. According to data from the
Federal Reserve Bank of St Louis, Federal tax revenue as a percentage of GDP
has remained constant for the last 70 years\autocite{revenueaspercentofgdp},
the highest marginal tax rate has reached as high as 90\% and has dipped as low
as 28\%\autocite{taxbrackets}. This phenomenon can largely be attributed to the
fact that high taxes lower the incentive to make money. On the other hand, when
tax rates are low, the incentive to innovate, take risks, and work hard is
returned and the economy is more productive. By lowering tax rates, total tax
revenue will not decrease significantly and all Americans will have more agency
over their own money.

During World War Two, America's 90\% tax bracket didn't bring in that much more
money. Because the high rates discouraged work, President Kennedy backed a bill
that lowered the top rate to 70\%\autocite{revenueactof1964}. During this time,
despite the 70\% rate, a millionaire on average paid 41\% of their income in
taxes. Instead of investing in new innovation, the rich invested in tax
shelters and accountants. Ronald Reagan collaborated with the Democrats to
lower the tax rate in exchange for consolidating the tax
code.\autocite{revenueactof1986} Our proposition is similar to that bipartisan
tax reform.

The reform proposed here is deficit-neutral. This means that the revenue gained
from closing loopholes will outweigh the cuts to tax rates. Specifically, we
propose to repeal or cut the following provisions: earned income tax credit,
research tax credit, energy tax preferences, state and local tax deduction, and
the mortgage interest deduction. In terms of tax rates, we suggest to cut the
corporate tax rate from 21\% to 15\%, reform the income tax rates from a
seven-rate structure to a bi-rate structure of 12\% and 25\%, and lessen the
capital gains and dividends tax rate from 23.8\% to 20\%.

Critics of this plan might argue that exemptions for environmentally friendly
behavior should remain. We believe that although those exemptions have good
intentions, the negative effects from artificially increasing demand in certain
sectors is severely detrimental to the economy. Some exemptions are beneficial,
though. For example, we plan to not change the charitable donation deduction.
The charitable donation deduction allows for individuals and companies to
support certified organizations. Unlike many other deductions, this is
accessible to the average person.

In conclusion, our plan will simplify the tax code, lower rates across the
board, and put more power in the hands of the American people.

\printbibliography[title={Bibliography}]

\end{document}
