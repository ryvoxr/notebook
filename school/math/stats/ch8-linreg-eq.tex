\documentclass{article}

\usepackage{booktabs}
\usepackage{amsmath}
\usepackage{array}
\usepackage{bm}
\usepackage{float} 
\usepackage[margin=1.5in]{geometry}

\newcolumntype{L}{>{$}l<{$}}

\title{Stats Thursday 9-21}
\author{Henry Oehlrich}

\begin{document}
\maketitle

\section*{1. Regression equations}
\centering
\begin{tabular}{l|L|L|L|L|L|L}
    \toprule
    ? & \bar{x} & S_x & \bar{y} & S_y & r & \hat{y} = \beta_0 + \beta_1x \\
    \midrule
    a & 10 & 2 & 20 & 3 & 0.5 & \bm{\hat{y} = 12.5 + 0.75x} \\
    b & 2 & 0.06 & 7.2 & 1.2 & -0.4 & \bm{\hat{y} = 23.2 -8x} \\
    c & 12 & 6 & \bm{152} & \bm{30} & -0.8 & \hat{y} = 200 - 4x \\
    d & 2.5 & 1.2 & \bm{25} & 100 & \bm{0.6} & \hat{y} = -100 + 50x \\
\end{tabular}
\raggedright

\subsection*{a.}
\begin{gather}
    \hat{y} = \beta_0 + \beta_1x \\
    \beta_1 = \frac{rs_y}{s_x} = \frac{0.5(3)}{2} = 0.75 \\
    \bar{y} = \beta_0 + \beta_1\bar{x} \\
    \beta_0 = \bar{y} - \beta_1\bar{x} = 20 - 0.75(10) = 12.5 \\
    \hat{y} = 12.5 + 0.75x
\end{gather}

\subsection*{b.}
\begin{gather}
    \hat{y} = \beta_0 + \beta_1x \\
    \beta_1 = \frac{rs_y}{s_x} = \frac{-0.4(1.2)}{0.06} = -8 \\
    \bar{y} = \beta_0 + \beta_1\bar{x} \\
    \beta_0 = \bar{y} - \beta_1\bar{x} = 7.2 - -8(2) = 23.2 \\
    \hat{y} = 23.2 -8x
\end{gather}

\subsection*{c.}
\begin{gather}
    \bar{y} = \beta_0 + \beta_1\bar{x} = 200 - 4(12) = 152 \\
    \beta_1 = \frac{rs_y}{s_x} \\
    s_y = \frac{\beta_1s_x}{r} = \frac{-4(6)}{-0.8} = 30 
\end{gather}

\subsection*{d.}
\begin{gather}
    \bar{y} = \beta_0 + \beta_1\bar{x} = -100 + 50(2.5) = 25 \\
    \beta_1 = \frac{rs_y}{s_x} \\
    r = \frac{\beta_1s_x}{s_y} = \frac{50(1.2)}{100} = 0.6
\end{gather}

\section*{5. Least squares}
\begin{table}[H]
    \begin{minipage}{.3\textwidth}
        \begin{tabular}{L|L|L|L}
            \toprule
            x & y & \hat{y} & SS \\
            \midrule
            10 & 10 & 18 & -8 \\
            20 & 50 & 29 & 21 \\
            40 & 20 & 51 & -31 \\
            50 & 80 & 61 & 19 \\
        \end{tabular}
    \end{minipage}
    \begin{minipage}{.7\textwidth}
        "Least squares" specifies the line that minimizes the variance of the
        residuals; that is, the line that minimizes the sum of the squared
        residuals. Given these data, the line of least squares is $\hat{y} =
        7.0 + 1.1x$. The general equation for the line of best fit is $\hat{y}
        - \beta_0 + \beta_1x$. We can solve for $\beta_0$ by setting $\hat{y} =
        \bar{y}$ and $x = \bar{x}$. The value of $\beta_1 = \frac{rs_y}{s_x}$
    \end{minipage}
\end{table}

\section*{9. More real estate}
$\widehat{price}=47.82+0.061size$

\subsection*{a.}
The slope of the line is 0.061. This means that for every square foot of space,
the price of the house increases by \$0.061.

\subsection*{b.}
\begin{gather}
    \widehat{price}=47.82+0.061size \\
    \widehat{price}=47.82+0.061(3000) \\
    \widehat{price}=47.82+183 \\
    \widehat{price}=230.82
\end{gather}

\subsection*{c.}
\begin{table}[H]
    \begin{minipage}{.6\textwidth}
        \begin{gather}
            \widehat{price}=47.82+0.061size \\
            \widehat{price}=47.82+0.061(1200) \\
            \widehat{price}=47.82+73.2 \\
            \widehat{price}=121.02 \\
            price = \widehat{price} - 6 = 115.02
        \end{gather}
    \end{minipage}
    \quad
    \begin{minipage}{.3\textwidth}
        The asking price is \$115,000. The predicted price is \$121,020. The
        difference of \$6,020 is called the residual.
    \end{minipage}
\end{table}

\section*{11. What slope?}
The slope is most likely to be 300. The slope is the change in weight for every
1 foot of length. A car's length in feet is likely to be in the range of 10-20.
10-20 times 3 is 30-60. 10-20 times 30 is 300-600. 10-20 times 300 is
3000-6000. 10-20 times 3000 is 30000-60000. The most reasonable of these ranges
is 3000-6000 created by a slope of 300.

\section*{17. SAT}

\subsection*{a.}
As the Verbal SAT score increases, the expected Math SAT score increases. The
scatter plot shows a moderately strong positive linear correlation. The
scatterplot shows moderate scatter and ~1 form outlier.

\subsection*{b.}
Yes, a student did perfectly on the Math section but did very poorly on the
Verbal section.

\subsection*{c.}
The correlation coefficient $r$ is 0.685. This means that there is a moderately
strong positive linear correlation between the Math and Verbal SAT scores.
46.9\% of variation in the Math SAT scores can be explained by Verbal SAT
scores.

\subsection*{d.}
\begin{table}[H]
    \begin{minipage}{.3\textwidth}
        \begin{tabular}{L|L}
            \toprule
            \midrule
            r & 0.685 \\
            \overline{verbal} & 596.3 \\
            s_{verbal} & 99.5 \\
            \overline{math} & 612.2 \\
            s_{math} & 96.1
        \end{tabular}
    \end{minipage}
    \begin{minipage}{.7\textwidth}
        \begin{gather}
            \widehat{math} = \beta_0 + \beta_1verbal \\
            \beta_1 = \frac{rs_{math}}{s_{verbal}} = \frac{0.685(96.1)}{99.5} = 0.662 \\
            \overline{math} = \beta_0 + \beta_1\overline{verbal} \\
            \beta_0 = \overline{math} - \beta_1\overline{verbal} = 612.2 - 0.662(596.3) = 217.449 \\
            \widehat{math} = 217.449 + 0.verbal
        \end{gather}
    \end{minipage}
\end{table}

\subsection*{e.}
The slope of the line of best fit is $\widehat{math} = 217.449 + 0.662verbal$.
This means that for every 1 point increase in the Verbal SAT score, the
predicted Math SAT score increases by 0.662 points. 

\subsection*{f.}
\begin{gather}
    \widehat{math} = 217.449 + 0.662verbal \\
    \widehat{math} = 217.449 + 0.662(500) = 548.449
\end{gather}

\subsection*{g.}
\begin{gather}
    \widehat{math} = 217.449 + 0.662verbal \\
    \widehat{math} = 217.449 + 0.662(800) = 747.049 \\
    math - \widehat{math} = 800 - 747.049 = 52.951
\end{gather}

\section*{19. SAT, take 2}

\subsection*{a.}
There would be a moderately strong positive linear correlation. As the Math SAT
score increases, the Verbal SAT score increases. The coefficient of correlation
for these data are 0.685. The coefficient of correlation does not change
because switching the explanatory and response variables does not change the
strength of the linear relationship.

\subsection*{b.}
\begin{table}[H]
    \begin{minipage}{.3\textwidth}
        \begin{tabular}{L|L}
            \toprule
            \midrule
            r & 0.685 \\
            \overline{verbal} & 596.3 \\
            s_{verbal} & 99.5 \\
            \overline{math} & 612.2 \\
            s_{math} & 96.1
        \end{tabular}
    \end{minipage}
    \begin{minipage}{.7\textwidth}
        \begin{gather}
            \widehat{verbal} = \beta_0 + \beta_1math \\
            \beta_1 = \frac{rs_{verbal}}{s_{math}} = \frac{0.685(99.5)}{96.1} = 0.709 \\
            \overline{verbal} = \beta_0 + \beta_1\overline{math} \\
            \beta_0 = \overline{verbal} - \beta_1\overline{math} = 596.3 - 0.709(612.2) = 162.392 \\
            \widehat{verbal} = 162.392 + 0.709math
        \end{gather}
    \end{minipage}
\end{table}

\subsection*{c.}
In this context, a positive residual would mean that a student scored higher on
the Verbal SAT than predicted by the regression equation. A negative residual
would mean that a student scored lower on the Verbal SAT than predicted by the
regression equation.

\subsection*{d.}
\begin{gather}
    \widehat{verbal} = 162.392 + 0.709math \\
    \widehat{verbal} = 162.392 + 0.709(500) = 516.892
\end{gather}
\end{document}

