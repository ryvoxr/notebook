\documentclass{article}

\title{Mini-project: NC Income and SAT}
\author{Henry Oehlrich}

\usepackage[margin=1.25in]{geometry}
\usepackage{pgfplots}
\usepackage{pgfplotstable}
\usepackage{pgf-pie}
\usepackage{caption}
\usepackage{subcaption}
\usepackage{booktabs}

\pgfplotsset{compat=1.18}
\usepgfplotslibrary{statistics}
\captionsetup[figure]{labelfont={bf}}

\pgfplotstableread{data.dat}\data
\pgfplotstableread{resid.dat}\resid
\pgfplotstableread{sample1.dat}\sampleone
\pgfplotstableread{sample2.dat}\sampletwo
\pgfplotstableread{sample3.dat}\samplethree
\pgfplotstableread{sample4.dat}\samplefour
\pgfplotstableread{sample5.dat}\samplefive

\begin{document}
\maketitle

\section{Regression Analysis}

\begin{figure}[h]
    \centering
    \begin{tikzpicture}
        \begin{axis}[
                xlabel={Voter Turnout},
                ylabel={SAT Score},
                width=\textwidth,
            ]
            \addplot +[only marks,mark size=1.5pt] table
            {\data};
            \addplot table [
                y={create col/linear regression={y=sat}},
                mark=none,
            ]
            {\data};
            \addlegendentry{SAT vs. Voter Turnout(t)}
            \addlegendentry{$\widehat{SAT} = \pgfmathprintnumber{\pgfplotstableregressionb} + \pgfmathprintnumber{\pgfplotstableregressiona} \cdot t$}
        \end{axis}
    \end{tikzpicture}
    \caption{Scatterplot}
    \label{fig:scatter}
\end{figure}

\begin{figure}[h]
    \begin{minipage}{0.7\textwidth}
        The slope of the regression line is $675.26$, which means that for
        every one percent increase in voter turnout, the predicted SAT score
        increases by $\pgfmathprintnumber{\pgfplotstableregressiona}$ points.
        The correlation coefficient is $0.386$, which shows that there is a
        weak positive correlation between voter turnout and SAT score. The
        coefficient of determination is $0.149$, which means that $14.9\%$ of
        the variation in SAT score can be explained by voter turnout.
    \end{minipage}%
    \hspace{0.05\textwidth}
    \begin{minipage}{0.2\textwidth}
        \centering
        \begin{tabular}{l|l}
            \toprule
            Statistic & Value \\
            \midrule
            $\beta_1$ & $\pgfmathprintnumber{\pgfplotstableregressiona}$ \\
            $r$ & $0.386$ \\
            $R^2$ & $0.149$ \\
            $\bar{t}$ & $0.599$ \\
            $\overline{SAT}$ & $1407.35$ \\
            \bottomrule
        \end{tabular}
    \end{minipage}
\end{figure}

\newpage

\begin{figure}[h]
        \centering
        \begin{tikzpicture}
            \begin{axis}[
                    xlabel={Voter Turnout},
                    ylabel={Residual},
                    width=0.8\textwidth,
                ]
                \addplot +[only marks,mark size=1.5pt] table
                {\resid};
            \end{axis}
        \end{tikzpicture}
        \caption{Residual Plot}
        \label{fig:resid}
\end{figure}

A linear model is appropriate for this data because the plot of the
explanatory variable against the response variable is \textit{straight
enough} (Figure \ref{fig:scatter}) and the residual plot is random and
has no pattern (Figure \ref{fig:resid}).

\newpage

\section{Sample Simulation}

Five samples of sample size 5 were collected from the data set. For each entry
of each sample, one county was randomly selected from the data set. Individual
regressions were performed on each sample, and the results are shown in Figure
\ref{fig:samples}.

\begin{figure}[h]
    \centering
    \begin{tikzpicture}
        \begin{axis}[
                width=\textwidth,
                height=0.55\textwidth,
                xlabel={Voter Turnout},
                xtick={0.3,0.4,...,0.7},
                ylabel={SAT Score},
                legend pos=south east,
            ]
            \pgfplotstablecreatecol[linear regression={y=sat},]{regression}{\sampleone}
            \xdef\slopeone{\pgfplotstableregressiona}
            \addplot[dashdotted,domain=0.3:0.6] {\pgfplotstableregressionb+\pgfplotstableregressiona*x};

            \pgfplotstablecreatecol[linear regression={y=sat},]{regression}{\sampletwo}
            \xdef\slopetwo{\pgfplotstableregressiona}
            \addplot[dashdotted,domain=0.3:0.6] {\pgfplotstableregressionb+\pgfplotstableregressiona*x};

            \pgfplotstablecreatecol[linear regression={y=sat},]{regression}{\samplethree}
            \xdef\slopethree{\pgfplotstableregressiona}
            \addplot[dashdotted,domain=0.3:0.6] {\pgfplotstableregressionb+\pgfplotstableregressiona*x};

            \pgfplotstablecreatecol[linear regression={y=sat},]{regression}{\samplefour}
            \xdef\slopefour{\pgfplotstableregressiona}
            \addplot[dashdotted,domain=0.3:0.6] {\pgfplotstableregressionb+\pgfplotstableregressiona*x};

            \pgfplotstablecreatecol[linear regression={y=sat},]{regression}{\samplefive}
            \xdef\slopefive{\pgfplotstableregressiona}
            \addplot[dashdotted,domain=0.3:0.6] {\pgfplotstableregressionb+\pgfplotstableregressiona*x};

            \pgfplotstablecreatecol[linear regression={y=sat},]{regression}{\data}
            \addplot[red,domain=0.3:0.6] {\pgfplotstableregressionb+\pgfplotstableregressiona*x};

            \addlegendentry{Sample 1}
            \addlegendentry{Sample 2}
            \addlegendentry{Sample 3}
            \addlegendentry{Sample 4}
            \addlegendentry{Sample 5}
            \addlegendentry{Population}
        \end{axis}
    \end{tikzpicture}
    \caption{Plot of Sample Regressions}
    \label{fig:samples}
\end{figure}

\begin{figure}[ht]
    \begin{minipage}{0.4\textwidth}
        Figure \ref{fig:samples} shows that the regression lines of the samples
        are all similar to each other in slope and intercent. Variance in the
        intercepts of the samples and the slope of the population is random (two
        lines above, three lines below). The histogram in Figure \ref{fig:hist}
        shows the distribution of the slopes of the sample regressions.
        Although their is no clear center (due to the small sample size), the
        slope of the population is in the center of the distrobution. 2 samples
        are to the left and 3 are to the right.
    \end{minipage}%
    \hspace{0.05\textwidth}
    \begin{minipage}{0.6\textwidth}
        \centering
        \begin{tikzpicture}
            \begin{axis}[
                    width=\textwidth,
                    height=0.6\textwidth,
                    xlabel={Slope},
                    ylabel={Frequency},
                    xtick=data,
                    ymin=0,
                    ybar,
                    tick align=outside,
                ]
                \addplot+[hist={bins=6,data min=200, data max=1400}]
                table [row sep=\\,y index=0] {
                    data\\
                    \slopeone\\
                    \slopetwo\\
                    \slopethree\\
                    \slopefour\\
                    \slopefive\\
                };
            \end{axis}
        \end{tikzpicture}
        \caption{Histogram of Slopes of Sample Regressions}
        \label{fig:hist}
    \end{minipage}
\end{figure}

\end{document}
