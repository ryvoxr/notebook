\documentclass{article}
\title{Fritz Haber and Population}
\author{Henry Oehlrich}
\begin{document}
\maketitle{}

In the article \textit{Fritz Haber: Jewish chemist whose work led to Zyklon B},
Chris Bowlby explores the history and effects of the life and work of Fritz
Haber. It was stated that "as many as two out of five humans on the planet
today owe their existence" to Fritz Haber. This is because, in the midst of a
food crisis due to a lack of fertilizer, Haber discovered a process to create
ammonia for fertilizer from nitrogen and hydrogen. As mentioned in previous
units, bird poop was the major source of ammonia for food for many years. His
method of producing the limiting factor of the food supply was described as
creating "bread from air". His process, aptly named the Haber-process, was also
used in both word wars for making explosives and poison gas. By creating the
opportunity for billions to live in the future, he also allowed for the
homicide of millions. It could be argued that Haber is responsible for the
current state of our world population and that his discovery might be the
single most important factor that will contribute to the possible severe crash
of the world population.

Haber's story exemplifies a powerful struggle in our current world relating to
population. As more and more countries become more developed and attain access
to better health infrastructure, the crude death rate in those countries will go
down (resulting in the population growth rate increasing). The boom of
population growth in the transitional period of developing countries
contributes majorly to the world population growth. The moral dilemma here is
that by preventing more people from dying now, it could be argued, that more
people will die later due to a population crash.

\end{document}
