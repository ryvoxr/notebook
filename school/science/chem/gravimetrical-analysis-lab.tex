\documentclass{article}
\usepackage{amsmath}
\usepackage{booktabs}

\begin{document}
\section{Gravimetrical Analysis of a Metal Carbonate}
\subsection{Calculations}
\begin{gather}
    {CaCO_3}_{mass} = CaCO_3\ (2nd) - filter\ paper = 3.79g - 1.73g = 2.06g \\
    {CaCO_3}_{moles} = \frac{{CaCO_3}_{mass}}{M_{CaCO_3}} = \frac{2.06g}{100.09\frac{g}{mol}} = 0.0206mol \\
    M_M = \frac{M_2CO_3}{CaCO_3} = \frac{2.00g}{0.0206mol} = 97.09\frac{g}{mol} \\
    \left( M_M = 97.09\frac{g}{mol} \right) \approx \left( M_{Na_2CO_3} = 105.99\frac{g}{mol} \right) \\
    M = Na_2CO_3
\end{gather}
\subsection{Percent Error}
\begin{gather}
    \frac{M_{Na_2CO_3} - M_{M_2CO_3}}{M_{Na_2CO_3}} \times 100\% = \frac{105.99\frac{g}{mol} - 97.09\frac{g}{mol}}{105.99\frac{g}{mol}} \times 100\% = 8.5\%
\end{gather}
\\
Our error of 14\% was due to the fact the \(CuSO_4\) is hygroscopic and reabsorbed a portion of the water that was evaporated.
\\\\
\(Cu(II)SO_4\cdot5H_2O \to Cu(II)SO_4 + 5H_2O\)
\\\\
\centering
\begin{tabular}{p{5cm}|p{2cm}}
    \toprule
    Mass & Value \\
    \midrule
    \(M_2CO_3\) & 2.00g \\
    Filter paper & 1.73g \\
    Filter paper and \(CaCO_3\) (1st) & 4.34g \\
    Filter paper and \(CaCO_3\) (2nd) & 3.79g \\
\end{tabular}
\end{document}
