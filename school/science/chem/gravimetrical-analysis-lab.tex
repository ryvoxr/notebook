\documentclass{article}
\usepackage{amsmath}
\usepackage{booktabs}

\begin{document}
\section{Gravimetrical Analysis of a Metal Carbonate}

\subsection{Purpose}

This lab was done to determine the identity of a Group 1 metal carbonate
compound by the process of gravimetrical analysis. This was done by the unknown
being weighed and then dissolved in water. Then \(CaCl_2\) was added to the
metal carbonate solution to precipitate \(CaCO_3\). The precipitate was then
filtered and dried. The mass of the precipitate was then measured and used to
determine the identity of the metal carbonate.

\subsubsection{Equation}
\(CaCl_2 + M_2CO_3 \rightarrow CaCO_3 + 2MCl\)

\subsection{Procedure}

The crucible was weighed and 2 grams of the mystery element and 200ml of water
was added to the beaker. To this, 125 ml of the 0.2 M \(CaCl_2\) was added.
This was then stirred with the glass rod and left alone for 5 minutes to set.
The mass of the filter paper was taken and recorded. After filtering the water
out, the precipitate was rinsed with deionized water to wash out any residual
salts other than \(CaCO_3\) and the remaining precipitate was taken and dried
in the microwave so that the water would evaporate. Finally, the dried up solid
was weighed once again and recorded a final time.

\subsection{Data}

\begin{tabular}{p{5cm}|p{2cm}}
    \toprule
    Mass & Value \\
    \midrule
    \(M_2CO_3\) & 2.00g \\
    Filter paper & 1.73g \\
    Filter paper and \(CaCO_3\) (1st) & 4.34g \\
    Filter paper and \(CaCO_3\) (2nd) & 3.79g \\
\end{tabular}

\subsubsection{Materials}

\begin{tabular}{p{5cm}|p{5cm}}
    \toprule
    \midrule
    125ml \(CaCO_3\) 0.2M & Filter funnel \\
    200ml water & Filter paper \\
    2g \(M_2CO_3\) & Glass stirring rod \\
    Analytical balance & 250 ml graduated cylinder \\
    2 400ml beakers & Ring stand and iron ring \\
    15 ml crucible & Wash bottle \\
    Microwave & Crucible tongs \\
    \bottomrule
\end{tabular}

\subsection{Calculations}

\begin{gather}
    {CaCO_3}_{mass} = CaCO_3\ (2nd) - filter\ paper = 3.79g - 1.73g = 2.06g \\
    {CaCO_3}_{moles} = \frac{{CaCO_3}_{mass}}{M_{CaCO_3}} = \frac{2.06g}{100.09\frac{g}{mol}} = 0.0206mol \\
    M_M = \frac{M_2CO_3}{CaCO_3} = \frac{2.00g}{0.0206mol} = 97.09\frac{g}{mol} \\
    \left( M_M = 97.09\frac{g}{mol} \right) \approx \left( M_{Na_2CO_3} = 105.99\frac{g}{mol} \right) \\
    M = Na_2CO_3
\end{gather}

\subsubsection{Percent Error}

\begin{gather}
    \frac{M_{Na_2CO_3} - M_{M_2CO_3}}{M_{Na_2CO_3}} \times 100\% = \frac{105.99\frac{g}{mol} - 97.09\frac{g}{mol}}{105.99\frac{g}{mol}} \times 100\% = 8.5\%
\end{gather}

\subsection{Conclusion}

We successfully met the purpose of this lab because we were able to identify
that the unknown Group 1 metal carbonate was \(Na_2CO_3\) with a relatively
small margin of error. What was likely the largest source of error in our
procedure was not fully evaporating all the water from filter paper and the
\(CaCO_3\). This would have resulted in measuring a greater mass of \(CaCO_3\).
We would then divide the grams of \(M_2CO_3\) by this larger number of moles
resulting in a smaller molar mass than the actual of 105.99 grams ultimately
causing our 8.5\% error.

\end{document}
