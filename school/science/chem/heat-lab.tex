\documentclass{article}

\usepackage{amsmath}
\usepackage{booktabs}
\usepackage{siunitx}

\begin{document}

\begin{tabular}{l|l|l|l}
    \toprule
    Mass of Ionic Solid & Mass of Water & Initial Temperature & Final Temperature \\
    \midrule
    0.5g & 44.44g & 23\textdegree C & 17.1\textdegree C \\
    0.5g & 39.24g & 21.9\textdegree C & 15.4\textdegree C \\
\end{tabular}

\begin{gather}
    q = mc\Delta T \\
    q_1 = (44.44\si{\gram} + 5\si{\gram})(4.184\frac{J}{g\cdot\si{\degreeCelsius}})(17.1\si{\degreeCelsius} - 23\si{\degreeCelsius}) = -1.22\si{\kilo\joule} \\
    q_2 = (39.24\si{\gram} + 5\si{\gram})(4.184\frac{J}{g\cdot\si{\degreeCelsius}})(15.4\si{\degreeCelsius} - 21.9\si{\degreeCelsius}) = -1.20\si{\kilo\joule} \\
    \Delta H = \frac{-q}{mol_{rxn}} \\
    mol_{rxn} = \frac{1\si{\mol}\; CO(N\!H_2)_2}{60.06\si{\gram}\; CO(N\!H_2)_2} \cdot 5.0\si{\gram}\; CO(N\!H_2)_2 = 0.0833\si{\mol} \\
    \Delta H_1 = \frac{q_1}{mol_{rxn}} = \frac{1.22\si{\kilo\joule}}{0.0833\si{\mol}\; CO(N\!H_2)_2} = 14.65\si{\kilo\joule\per\mol} \\
    \Delta H_2 = \frac{q_1}{mol_{rxn}} = \frac{1.20\si{\kilo\joule}}{0.0833\si{\mol}\; CO(N\!H_2)_2} = 14.41\si{\kilo\joule\per\mol} \\
    \Delta H = \frac{H_1 + H_2}{2} = \frac{14.65 + 14.41}{2} = 14.53\si{\kilo\joule\per\mole} \\
    \delta = \frac{|V - V_m|}{V} = \frac{15.39 - 14.53}{15.39} \cdot 100\% = 5.6\%
\end{gather}

\end{document}
