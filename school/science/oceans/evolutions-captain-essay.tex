\documentclass{article}
\title{Technology's Role in Research Over Time}
\author{Henry Oehlrich}
\begin{document}
\maketitle{}

The role of technology over the last three centuries has fundamentally changed
how research is conducted. The book \textit{Evolution's Captain}, by Peter
Nichols, exemplifies how technology was used in the 19th century by exploring a
section of the research conducted by Charles Darwin and Robert Fitz Roy. The 
difference in the manner that science was conducted between the 19th century
and now shows how research has become less hands on, safer, and more analysis
focused.

Technology has become progressively less hands on over time. Because of this,
modern research is almost all done on the computer—either analyzing data or
conducting simulations. In \textit{Evolution's Captain}, Fitz Roy and the crew
of the \textit{HMS Beagle} are actually on a boat, conducting the research in
person. This raw interface with the research requires them to be more
knowledgeable about practical things. For example, when they were in need of a
boat, they ``fashioned something resembling an Irish coracle, a wicker like
intertwining of branches covered with pieces of canvas cut from their tent, the
inside packed with dense, clayed dirt.'' Modern cartographers wouldn't be able
to do that. Nowadays, there is a layer of abstraction and protection from the
work—lives are generally not in danger to create a map.

Modern technology has eliminated the need for people to map most things. This
is primarily because the surface of the earth is all but completely charted.
GPS technology has allowed for extremely precise location detection, and
satellites gives insight on the live surface features of the earth with stunning
accuracy. Compared to the clunky equipment from the time period of the book,
this technology is paradigm shifting. Even with all of the fancy tech, it is
still sometimes necessary to be on-site. For example, even now, one of the only
reasonable ways to get a high detail map of the ocean is to go out with a boat
and some scanning equipment.

Over time, a shift from discovery to analysis is apparent in science. As of now,
many things that are easy to discover or find, already are (we know where are of
the continents and countries are). The evolution of the World Wide Web allows
scientists to access vast amounts of (generally unbiased) data on the world.
Whereas Charles Darwin and his team were making the first semi-accurate map of
Tierra del Fuego, modern researchers might be investigating and observing the
effects of the introduction of beavers becoming invasive there.

\end{document}
