\documentclass{article}

\usepackage{tikz}
\usepackage{circuitikz}
\usepackage[labelfont=bf]{caption}
\usepackage{booktabs}
\usepackage{caption}
\usepackage{siunitx}
\usepackage{amsmath}
\usepackage{pgfplots}
\usepackage{pgfplotstable}
\usepackage{subcaption}
\usepackage[margin=1.5in]{geometry}

\title{Proving Ohm's Law using a Parallel and Series Circuit}
\author{Henry Oehlrich\and Sravya Potluru\and Chelsea Liao\and Owen Yao}

\begin{document}
\maketitle

\section{Abstract}

The purpose of this experiment is to prove Ohm's Law using a parallel and
series circuit. This was done by varying the voltage and measuring the current.
This procedure was done independently for both circuits. This data can then be
used to calculate the experimental equivalent resistances of the circuits. We
proved Ohm's Law by showing that current varied with voltage in a manner
consistent with Ohm's law and the resistance of the circuit.

\section{Materials and Methods}

\subsection{Materials}
\begin{itemize}
    \item Variable power supply (with current and voltage meters built in). We
        chose this because it allowed us to vary the voltage while
        automatically measuring the current in the circuit. This allowed us to
        reduce the complexity of the experiment setup and focus on data
        collection.
    \item 2 \SI{10}{\ohm} resistors. We chose \SI{10}{\ohm} resistors because
        they produced a current that was within the measurement range of the
        ammeter.
    \item Alligator clips. We chose these because we found that these were the
        easiest way to connect the circuit components.
\end{itemize}


\subsection{Part One (Series Circuit)}
\begin{enumerate}
    \item Connect the circuit components as shown in Figure 1 using
        alligator clips.
    \item Set the power supply to \SI{2}{\volt} and record the current in the
        circuit.
    \item Increase the voltage by \SI{0.5}{\volt} and record the current.
        Repeat this step until the voltage reaches \SI{5}{\volt}. 
\end{enumerate}

This circuit was simple enough that there was no need to use a breadboard.  We
chose the data collection range of \SI{2}{\volt} to \SI{5}{\volt} because the
current created was within the measurement range of the ammeter.

\subsection{Part Two (Parallel Circuit)}
\begin{enumerate}
    \item Connect the circuit components as shown in Figure 2 using
        alligator clips
    \item Set the power supply to \SI{0.5}{\volt} and record the current in the
        circuit.
    \item Increase the voltage by \SI{0.5}{\volt} and record the current. Repeat
        this step until the voltage reaches \SI{3.5}{\volt}.
\end{enumerate}

We chose the data collection range of \SI{0.5}{\volt} to \SI{3.5}{\volt} in
order to create currents that within the measurement range of the ammeter and
were similar to the currents created in the series circuit.

% Circuit diagram here
\begin{figure}[h]
    \centering
    \begin{minipage}[b]{0.4\textwidth}
        \begin{circuitikz}
            \draw (0,0)
            to[battery, l=$V$, invert] (0,2)
            to[ammeter] (4,2)
            to[short] (4,0)
            to[R, l=$R$] (2,0)
            to[R, l=$R$] (0,0);
        \end{circuitikz}
        \label{fig:series}
        \caption{Series Circuit}
    \end{minipage}%
    \begin{minipage}[b]{0.4\textwidth}
        \begin{circuitikz}
            \draw (0,0)
            to[battery, l=$V$, invert] (0,2)
            to[ammeter] (2,2)
            to[R, l=$R$] (2,0)
            to[short] (0,0)
            (2,2) to[short] (4,2)
            to[R, l=$R$] (4,0)
            to[short] (2,0);
        \end{circuitikz}
        \label{fig:parallel}
        \caption{Parallel Circuit}
    \end{minipage}
\end{figure}

\section{Data Collected}

\begin{figure}[h]
    \centering
    \begin{subfigure}{0.4\textwidth}
        \begin{tabular}{ll}
            \toprule
            {Voltage (\si{\volt})} & {Current (\si{\ampere})} \\
            \midrule
            2.0 & 0.1 \\
            2.5 & 0.12 \\
            3.0 & 0.15 \\
            3.5 & 0.18 \\
            4.0 & 0.2 \\
            4.5 & 0.23 \\
            5.0 & 0.25 \\
            \bottomrule
        \end{tabular}
        \caption{Series Circuit}
    \end{subfigure}%
    \begin{subfigure}{0.4\textwidth}
        \begin{tabular}{ll}
            \toprule
            {Voltage (\si{\volt})} & {Current (\si{\ampere})} \\
            \midrule
            0.5 & 0.09 \\
            1.0 & 0.18 \\
            1.5 & 0.27 \\
            2.0 & 0.37 \\
            2.5 & 0.46 \\
            3.0 & 0.56 \\
            3.5 & 0.66 \\
            \bottomrule
        \end{tabular}
        \caption{Parallel Circuit}
    \end{subfigure}
    \caption{Data tables}
\end{figure}

We chose to collect data in the range of \SI{2}{\volt} to \SI{5}{\volt} for the
series circuit because the current was within the measurement range of the
ammeter. We chose to collect data in the range of \SI{0.5}{\volt} to
\SI{3.5}{\volt} for the parallel circuit because the current was within the
measurement range of the ammeter and was similar to the current created in the
series circuit. We chose to collect in increments of \SI{0.5}{\volt} because it
allowed us to collect a sufficient amount of data while keeping the experiment
time reasonable.

\newpage

\section{Graphs}

\begin{figure}[h]
    \centering
    \begin{tikzpicture}
        \begin{axis}[
                xlabel={Current (\si{\ampere})},
                ylabel={Voltage (\si{\volt})},
                width=0.7\textwidth,
                xmin=0.05,xmax=0.7,
                ymax=5.75,
                domain=0.1:0.25,
                legend style={at={(0.98,0.24)},anchor=north east}
            ]
            \addplot +[only marks,mark size=1.5pt] table
            {partone.dat};
            \addplot table [
                y={create col/linear regression={y=V}},
                mark=none,
            ]
            {partone.dat};
            \addlegendentry{$\widehat{V_{series}} = \pgfmathprintnumber{\pgfplotstableregressiona}I + \pgfmathprintnumber{\pgfplotstableregressionb}$}
        \end{axis}
        \begin{axis}[
                xlabel={Current (\si{\ampere})},
                ylabel={Voltage (\si{\volt})},
                width=0.7\textwidth,
                xmin=0.05,xmax=0.7,
                ymax=4,
                domain=0.09:0.66,
                legend style={at={(0.98,0.12)},anchor=north east}
            ]
            \addplot +[only marks,mark size=1.5pt,mark options={fill=green}] table
            {parttwo.dat};
            \addplot table [
                y={create col/linear regression={y=V}},
                mark=none,
            ]
            {parttwo.dat};
            \addlegendentry{$\widehat{V_{Parallel}} = \pgfmathprintnumber{\pgfplotstableregressiona}I + \pgfmathprintnumber{\pgfplotstableregressionb}$}
        \end{axis}
    \end{tikzpicture}
    \caption{Graph of Voltage vs. Current for Series and Parallel Circuits}
\end{figure}

\setlength{\jot}{10pt}
\begin{figure}
    \begin{gather}
        R_{eq} = R + R = \SI{10}{\ohm} + \SI{10}{\ohm} = \SI{20}{\ohm} \\
        V = I \cdot R_{exp} \\
        m = R_{exp} = \SI{19.38}{\ohm} \\
        error = \frac{R_{exp} - R_{eq}}{R_{eq}} \times 100 = \frac{19.38 - 20}{20} \times 100 = 3.1\%
    \end{gather}
    \caption{Series Circuit Calculations}
\end{figure}
\begin{figure}
    \begin{gather}
        R_{eq} = \frac{1}{\frac{1}{R} + \frac{1}{R}} = \frac{1}{\frac{1}{10} + \frac{1}{10}} = \SI{5}{\ohm} \\
        V = I \cdot R_{exp} \\
        m = R_{exp} = \SI{5.26}{\ohm} \\
        error = \frac{R_{exp} - R_{eq}}{R_{eq}} \times 100 = \frac{5.26 - 5}{5} \times 100 = 5.2\%
    \end{gather}
    \caption{Parallel Circuit Calculations}
\end{figure}

From Ohm's law, we know that the slope of the line of I plotted against V is
the equivalent resistance of the circuit. From the regressions on the data, we
found that the experimental resistance for the series circuit was
\SI{19.38}{\ohm} and the experimental resistance for the parallel circuit was
\SI{5.26}{\ohm}. The percent error for the series circuit was 3.1\% and the percent error for
the parallel circuit was 5.2\%. Because the percent error was less than 10\%
(the maximum tolerance of the resistors), we can conclude that Ohm's law is valid.

\section{Discussion}

We achieved our goal of proving Ohm's law using a parallel and series circuit.
Initially, we started with a \SI{100}{\kilo\ohm} resistor, but we found that
the current produced was not measurable. After further experimentation, we
found that a \SI{10}{\ohm} resistor produced a current that was within the
measurement range of the ammeter. We would improve this experiment by using a
more precise ammeter to reduce the error in our measurements. We would also
take the data three times (with three different resistors) in order to reduce
the variability caused by resistor tolerance.

\end{document}
