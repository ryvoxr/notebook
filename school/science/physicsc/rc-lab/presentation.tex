\documentclass{article}

\usepackage{tikz}
\usepackage[labelfont=bf]{caption}
\usepackage{booktabs}
\usepackage{caption}
\usepackage{siunitx}
\usepackage{amsmath}
\usepackage{pgfplots}
\usepackage{pgfplotstable}
\usepackage[bottom=1in]{geometry}

\title{Modeling the Time Constant of an RC Circuit}
\author{Henry Oehlrich\and Grace Jiang\and Jacob Wison\and Akhil Surapaneni}

\begin{document}
\maketitle

\section{Abstract}

\section{Procedure}

\begin{itemize}
    \item Insert the magnetic field sensor inside center of the solenoid with
        the white dot facing down the solenoid's long axis.
    \item Zero the magnetic field sensor.
    \item Measure the magnetic field at the center of the solenoid for currents
        \SI{0.0}{\ampere} to \SI{2.0}{\ampere} in \SI{0.5}{\ampere} increments.
    \item Plot the graph of magnetic field versus current
\end{itemize}

\section{Calculations and Graph}

\begin{tikzpicture}
    \begin{axis}[
            xlabel={Time (\si{\second})},
            ylabel={Potential (\si{\volt})},
            width=\textwidth,
            xmin=0, xmax=0.3,
            domain=0:0.3,
        ]
        \addplot{1.736^{-1.997*x}+4.496e-2};
    \end{axis}
\end{tikzpicture}

\setlength{\jot}{10pt}
\begin{gather}
    B = \mu_0 n I \\
    y = mx + b \\
    m = \mu_0 I \\
    \mu_0 = \frac{m}{I} \\
    m = \frac{\SI{1.6e-3}{\milli\tesla\cdot\meter}}{1.0}
    \left(\frac{\SI{1}{\tesla}}{\SI{1000}{\milli\tesla}}\right) = \SI{1.6e-3}{\tesla\cdot\meter} \\
    I = \SI{2}{\ampere} \\
    \mu_0 = \frac{\SI{1.6e-6}{\tesla\cdot\meter}}{\SI{2}{\ampere}} = \SI{8.0e-7}{\tesla\meter\per\ampere} \\
    err = \frac{4\pi \times 10^{-7}-8.0\times 10^{-7}}{4\pi \times 10^{-7}} \times 100\% = 36.3\%
\end{gather}

\newpage

\section{Discussion}

From our data, we calculated that the permeability of free space is
\SI{8.0e-7}{\tesla\meter\per\ampere}. This value is 36.3\% off from the
accepted value of $4\pi\times$\SI{e-7}{\tesla\meter\per\ampere}.

During the experiment, we encountered some difficulties. The final few coils of
the slinky were touching and shorting out the circuit. We fixed this by taping
the last few coils on both sides together. Additionally, when stretching the
coil to a length of \SI{1}{\meter} or greater, the tape holding the coil failed
and the coil retracted back to its original length. We adressed this by holding
onto the ends of the tape while stretching the coil (making sure not to touch
the coil itself and interfere with the data collection).

\end{document}

