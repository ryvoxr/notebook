\documentclass{article}

\usepackage{tikz}
\usepackage[labelfont=bf]{caption}
\usepackage{booktabs}
\usepackage{caption}
\usepackage{siunitx}
\usepackage{amsmath}
\usepackage{pgfplots}
\usepackage{pgfplotstable}
\usepackage[bottom=1in]{geometry}

\title{The Magnetic Moment of a Permanent Magnet}
\author{Henry Oehlrich\and Sravya Potluru\and Suhani Taylor\and Chelsea Liao}

\begin{document}
\maketitle

\section{Abstract}

The objective of this experiment is to determine the relationship between
magnetic field and current, determine the relationship between magnetic field
and the turn density of a solenoid, and to calculate an experimental value of
$\mu_0$. This will be done by varying the current and coil density and
measuring the resulting magnetic field.

\section{Procedure}

\subsection{How is the Magnetic Field in a Solenoid Related to Current?}

\begin{itemize}
    \item Insert the magnetic field sensor inside center of the solenoid with
        the white dot facing down the solenoid's long axis.
    \item Zero the magnetic field sensor.
    \item Measure the magnetic field at the center of the solenoid for currents
        \SI{0.0}{\ampere} to \SI{2.0}{\ampere} in \SI{0.5}{\ampere} increments.
    \item Plot the graph of magnetic field versus current
\end{itemize}

\subsection{How is the Magnetic Field in a Solenoid Related to the Spacing of the Turns?}

\begin{itemize}
    \item Insert the magnetic field sensor inside the center of the solenoid
        with the white dot facing down the solenoid's long axis.
    \item Zero the magnetic field sensor.
    \item Measure the magnetic field at the center of the solenoid for coil
        lengths \SI{0.5}{\meter} to \SI{2.0}{\meter} in \SI{0.5}{\meter} increments.
    \item Plot the graph of magnetic field versus coil density
\end{itemize}

\newpage

\section{Calculations and Graph}

\begin{figure}[h]
    \centering
    \begin{tikzpicture}
        \begin{axis}[
                xlabel={Current (\si{\ampere})},
                ylabel={Magnetic Field (\si{\milli\tesla})},
                y label style={at={(axis description cs:-0.02,.5)},anchor=south},
                width=0.8\textwidth,
                ymin=0,ymax=0.28,
                range=0:0.28,
            ]
            \addplot +[only marks,mark size=1.5pt] table
            {partone.dat};
            \addplot table [
                y={create col/linear regression={y=B}},
                mark=none,
            ]
            {partone.dat};
            \addlegendentry{$\hat{B} = \pgfmathprintnumber{\pgfplotstableregressiona}I + \pgfmathprintnumber{\pgfplotstableregressionb}$}
        \end{axis}
        \begin{axis}[
                xlabel={Coil desnity (\si{\per\meter})},
                xlabel near ticks,
                xmin=20,xmax=160,
                domain=20:160,
                axis x line*=top,
                width=0.8\textwidth,
                ytick=\empty,
                yticklabels={},
                ymin=0,ymax=0.35,
                range=0:0.35,
                legend style={at={(0.98,0.89)},anchor=north east}
            ]
            \addplot +[only marks,mark size=1.5pt,mark options={fill=green}] table
            {parttwo.dat};
            \addplot table [
                y={create col/linear regression={y=B}},
                mark=none,
            ]
            {parttwo.dat};
            \addlegendentry{$\hat{B} = \pgfmathprintnumber{\pgfplotstableregressiona}n + \pgfmathprintnumber{\pgfplotstableregressionb}$}
        \end{axis}
    \end{tikzpicture}
\end{figure}

\setlength{\jot}{10pt}
\begin{gather}
    B = \mu_0 n I \\
    y = mx + b \\
    m = \mu_0 I \\
    \mu_0 = \frac{m}{I} \\
    m = \frac{\SI{1.6e-3}{\milli\tesla\cdot\meter}}{1.0}
    \left(\frac{\SI{1}{\tesla}}{\SI{1000}{\milli\tesla}}\right) = \SI{1.6e-3}{\tesla\cdot\meter} \\
    I = \SI{2}{\ampere} \\
    \mu_0 = \frac{\SI{1.6e-6}{\tesla\cdot\meter}}{\SI{2}{\ampere}} = \SI{8.0e-7}{\tesla\meter\per\ampere} \\
    err = \frac{4\pi \times 10^{-7}-8.0\times 10^{-7}}{4\pi \times 10^{-7}} \times 100\% = 36.3\%
\end{gather}

\newpage

\section{Discussion}

From our data, we calculated that the permeability of free space is
\SI{8.0e-7}{\tesla\meter\per\ampere}. This value is 36.3\% off from the
accepted value of $4\pi\times$\SI{e-7}{\tesla\meter\per\ampere}.

During the experiment, we encountered some difficulties. The final few coils of
the slinky were touching and shorting out the circuit. We fixed this by taping
the last few coils on both sides together. Additionally, when stretching the
coil to a length of \SI{1}{\meter} or greater, the tape holding the coil failed
and the coil retracted back to its original length. We adressed this by holding
onto the ends of the tape while stretching the coil (making sure not to touch
the coil itself and interfere with the data collection).

\end{document}
