\documentclass{article}

\usepackage{tikz}
\usepackage[labelfont=bf]{caption}
\usepackage{booktabs}
\usepackage{caption}
\usepackage{siunitx}
\usepackage{amsmath}
\usepackage{pgfplots}
\usepackage{pgfplotstable}

\title{The Magnetic Moment of a Permanent Magnet}
\author{Henry Oehlrich\and Sravya Potluru\and Suhani Taylor\and Chelsea Liao}

\begin{document}
\maketitle

\section{Abstract}

\section{Procedure}

\begin{itemize}
\end{itemize}

\section{Calculations and Graph}

\setlength{\jot}{15pt}
\begin{gather}
    B_{axis} = \frac{\mu_0}{4\pi} \frac{2\mu}{d^3} \\
    \hat{B_{axis}} = a \cdot d^{b} = 4.39e-6 \cdot d^{-3.499} \\
    a = \frac{\mu_0 2 \mu (10^3)}{4\pi} \\
    \mu =
    \frac{\SI{4.39e-6}{\ampere\tesla\meter\cubed}}{\SI{10e-7}{\tesla\meter\per\ampere}(10^3)} = \SI{4.39e-3}{\ampere\meter\squared}
\end{gather}

\begin{center}
\begin{tikzpicture}
    \begin{axis}[
            xlabel={Distance (\si{\meter})},
            ylabel={Magnetic Field (\si{\milli\tesla})},
            width=\textwidth,
        ]
        \addplot +[only marks,mark size=1.5pt] table
        {data.dat};
        \addplot gnuplot [raw gnuplot, id=fit, no marks]{
            f(x) = a*x^(-b);
            a = 4.388e-6;
            b = 3.499;
            fit f(x) 'data.dat' using 1:2 via a,b;
            plot [x=0.0325:0.07] f(x);
        };
        \addlegendentry{$\hat{B_{axis}} = \num{2.942e-6} \cdot d^{-3.63}$}
    \end{axis}
\end{tikzpicture}
\end{center}

\section{Discussion}

We found that the magnetic field strength varied inversely with the 3.63 power.
This is 21\% off from the expected power for a true dipole of 3. We used the
power fit to find the magnetic moment of the magnet to be
\SI{4.39e-3}{\ampere\meter\squared}. When setting up the experiment, we found
that the magnets rolled off the meter stick. We remedied this by putting the
magnets on either side of the protractor. The perpendicular face of the
protractor prevented the magnets from rolling away.

\end{document}
